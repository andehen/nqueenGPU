\documentclass{article}
\usepackage[utf8]{inputenc}
\usepackage{amsmath,calc}
\usepackage{amssymb}
\usepackage{textcomp}
\usepackage{hyperref}

\title{Math-454 Parallel Computing and Pthreads \\
		Solving $n$-queens puzzle on GPU}
\author{Anders Asheim Hennum}
\date{June 8th}

\begin{document}

\maketitle

\begin{abstract}
Using GPU for scientific computing has become very popular. For many problems, 
the massive parallelism that GPU offers, can give magnificent speedups compared to 
normal CPU's. In this paper I have investigated if GPU is usable for solving puzzles. 
TODO: Rewrite when project is done. Add summarizing result.
\end{abstract}

\section{Introduction}

The puzzle I will try to solve is the $n$-queens puzzle \cite{nqueen}. It is a simple
and well documented puzzle. The puzzle is to place $n$ queens on an $n \times n$ chess
board without any queen attacking another queen. The approach I will use is a state-space
search with backtracking. This approach relies on random searches with very variable running times.
By using GPU, we can run multiple searches in parallel and by then, have a greater chance of finding
a solution faster than on a CPU. 

\section{The Algorithm}

TODO: Write pseudo code for the general algorithm. Explain.

\section{Implementation on GPU}

TODO: How did you implement it on GPU? Differences from general algorithm? Issues?
Memory, etc.? Random number generation issues?

\section{Results}

TODO: Add some nice graphs and tables with running times of CPU vs GPU.

\section{Conclusion}

TOOD: CPU vs GPU best when solving puzzles with state-space search approach? Advantages, 
disadvantages? What needs to be done to solve this kind of problems on GPU's?

\begin{thebibliography}{9}

\bibitem{nqueen}
	\url{http://en.wikipedia.org/wiki/Eight_queens_puzzle}

\end{thebibliography} 

\end{document}